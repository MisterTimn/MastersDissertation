\chapter{Technische aspecten}
\section{Machine Learning}
\npar Leren is een veelzijdig fenomeen dat bestaat uit  verschillende processen: het verkrijgen van declaratieve kennis, het ontwikkelen van motorische en cognitieve vaardigheden door instructie en ervaring, het organiseren van nieuwe kennis in algemene representaties en het ontdekken van nieuwe feiten via observatie en experimentatie.
\npar Sinds het begin van het computertijdperk proberen onderzoekers het menselijk leren na te bootsen en deze processen te vertalen naar de informatietheorie. Het machinaal leren is nog steeds een erg uitdagend doel in de kunstmatige intelligentie (KI).
\npar We kunnen zeggen dat een computerprogramma of machine leert als het zijn performantie op een bepaalde taak verbetert met ervaring \cite{machine_overview}
\npar In het geval van dit onderzoek is deze taak het classificeren van gebaren. Het model zal dus leren 
\section{Artifici\"eel neuraal netwerk}

\section{Convolutioneel neuraal netwerk}
\section{Hyperparameters}