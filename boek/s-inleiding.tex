\chapter{Inleiding}
\section{Gebarentaal}

Gebarentaal is in de eerste plaats een taal. Taal is een begrip dat moeilijk te defini\"eren valt en de meeste pogingen hiertoe beperken zich tot gesproken taal. Een definitie die ook voor gebarentaal kan gebruikt worden is: 
\\``Een taal is een natuurlijk ontstaan menselijk communicatiemidddel waarmee je kan communiceren over alles wat je denkt, ziet, voelt en droomt. Een taal bestaat uit bouwstenen. Die bouwstenen worden volgens bepaalde regels samengevoegd tot grotere gehelen. Elke taal heeft eigen bouwstenen en regels'' \cite{buyens_gebarentaaltolken_2003}.

\npar Gesproken taal en gebarentaal verschillen in de manier waarop gecommuniceerd wordt: oraal-auditief tegenover gestueel-visueel. Door middel van hand-, hoofd- en armbewegigingen wordt een woord ``uitgesproken'' en vervolgens visueel waargenomen.

\npar Een gebarentaal onstaat, net zoals een gesproken taal, spontaan en natuurlijk door contact tussen mensen. Net door deze spontane ontwikkeling is er geen universele gebarentaal. Evenals we verschillende gesproken talen en dialecten kennen per land of regio zijn er ook verschillende gebarentalen \cite{VGT-standard}. In Nederland is er bijvoorbeeld de Nederlandse Gebarentaal (NGT) en in Belgi\"e de Vlaamse Gebarentaal (VGT) en de Waalse Gebarentaal (la Langue des Signes de Belgique Francophone, LSFB). VGT verschilt dan weer van provincie tot provincie, met de grootste verschillen tussen West-Vlaanderen en Limburg, de twee verst uiteenliggende regio's.

\npar Een gebarentaal heeft een eigen grammatica en lexicon. Het lexicon of de gebarenschat is de verzameling van alle woorden of gebaren in de taal. Het lokale gebarenschat moet volledig onafhankelijk van het lokale woordenschat worden beschouwd.
\\Bepaalde woorden uit de ene taal kunnen niet eenduidig vertaald worden in een andere taal. Het woord "gezelligheid" kent bijvoorbeeld geen Engelse vertaling en voor het Duitse "fingerspitzengef\"uhl" hebben we in de Nederlandse taal ook geen alternatief.
\\Tussen een gebarentaal en een gesproken taal geldt dezelfde verhouding. Er is niet altijd een een-op-een relatie tussen een woord en een gebaar.

\npar Communicatie tussen doven en horenden is vaak een struikelblok. Sommige doven kunnen liplezen en zo opmaken wat een spreker wil vertellen. Voorwaarde hierbij is dat de spreker goed moet articuleren en natuurlijk niet te snel spreekt.
\\ Er kan ook altijd schriftelijk gecommunciceerd worden maar dit is een erg trage en onpersoonlijke vorm van communicatie. Ook is de bedrevenheid van een dove persoon in het schrijven van een gesproken taal vaak lager dan die van een horende.
\\ Doven kunnen zich ook beroepen op een tolk. Dit kan een vriend zijn die horende is en gebarentaal kent of een beroepstolk. In Vlaanderen kunnen doven terecht bij het Vlaams Communicatie Assistentie Bureau voor Doven (CAB) om een tolk in te huren. \cite{tolkuren} De Vlaamse overheid betaalt een aantal tolkuren terug. Onder andere achttien tolkuren voor priv\'edoeleinden, achttien voor sollicitaties en een situatie-afhankelijk aantal tolkuren voor arbeid en beroepsopleiding.
 
\section{Automatische gebarentaalherkenning}
Er is dus een communicatieprobleem tussen doven en horenden omdat ze niet dezelfde taal spreken. Er zijn ook vele verschillende gebarentalen en dialecten waardoor er tussen doven onderling ook niet altijd vlot gecommuniceerd wordt.
Door het gebruik van hedendaagse technologie moet het mogelijk zijn hierin te helpen en een automatisch herkenningssysteem uit te werken waarmee gebaren in real-time kunnen vertaald worden.

\npar Het herkennen van objecten of gebaren is iets waar de mens niet bij stilstaat. Een pasgeboren kind begint vanaf het openen van de ogen zijn waarneming en herkenningsvermogen te trainen. Terwijl we leren organiseren we vormen, objecten en categori\"en in nuttige taxonomi\"en en linken deze dan later naar onze taal \cite{oneshot-object-cat}. Eenmaal de leeftijd van zes jaar bereikt is kan een kind bijvoorbeeld 104 objectcategori\"en onderscheiden zonder hierbij stil te staan.

\npar Als mens kunnen we gebaren makkelijk differenti\"eren door registratie van armbewegingen, mimiek, houding van de handen en de manier waarop vingers gestrekt of geplooid worden. De neurologische fenomenen die deze vaardigheden kunnen verklaren worden nog steeds onderzocht. 

\npar Een machine of computer kan zien via het gebruik van een camera. Een beeld wordt voorgesteld door een matrix met pixelwaarden die de lichtintensiteit op dat bepaalde punt weergeeft. Traditioneel zijn er grijswaarden- en kleurbeelden maar tegenwoordig wordt ook vaak gebruikt gemaakt van 3D-cameratechnologi\"en, zoals de Microsoft Kinect \cite{kuhn2011kinect}, zodat er een aanvullend dieptebeeld is. Deze beelden gelden dan als de visuele data voor het systeem, daarna moeten specifieke technologi\"en worden ingezet om nuttige informatie uit deze data te halen.

\npar Een automatisch herkenningssysteem zal moeten leren omgaan met de grote variabiliteit van de invoer. De gebaren die het moet herkennen zullen uitgevoerd worden door mensen van verschillende grootte en lichaamsbouw. De vlotheid van het gebaren tussen ervaren en beginnende gebarentaligen zal sterk verschillen en de persoon zal niet altijd mooi recht in het midden van het beeld staan of even ver van de camera. Ook links- en rechtshandigheid heeft een invloed op het gebaren evenals de expressiviteit van de spreker.
\\ De aanwezigheid van andere mensen of veel beweging in de achtergrond bemoeilijkt ook het herkennen van gebaren. Daarenboven moet ook nog rekening gehouden worden met de lokale belichting. De spreker kan onderbelicht of overbelicht zijn waardoor bepaalde contouren moeilijker te detecteren vallen.

\npar Een compleet gebarentaalherkenningssysteem zal moeten voorzien in gebarensegmentatie, gebarenherkenning en grammaticale samenstelling van gebaren.

\subsection{Gebarensegmentatie}
\npar Wanneer we een persoon die gebarentaal spreekt registreren met een camera krijgen we een continue stroom aan informatie. In een bepaalde tijdspanne kan een persoon een of meerdere gebaren uitvoeren en het is onbekend waneer een gebaar begint of eindigt. De segmentatie van deze gebaren is dus een eerste uitdaging voor een herkenningssysteem. Er is minder belangstelling naar deze ``continue'' gebaarherkenningssystemen omdat vaak wordt uitgegaan van vooraf gesegmenteerde beelden \cite{hmdb-manual-segm}.
\npar  Tussen elk gebaar zit er een beweging die de overgang vormt tussen twee gebaren: de bewegingsepenthesis. Armen en handen gaan van eindpositie van het eerste gebaar naar beginpositie van het volgende. \cite{movement-epenthesis} Deze beweging moet gedetecteerd en gefilterd worden willen we een foutloze segmentatie krijgen.

\subsection{Gebarenherkenning}
Eenmaal we weten wanneer een gebaar begint en eindigt kunnen we het gaan identificeren. Uit de verzamelde visuele data wordt nuttige informatie ge\"extraheerd waarmee het model kan beslissen over welk gebaar het gaat. Het beeld wordt omgezet in een beeldrepresentatie, bestaande uit een of meerdere featurevectoren. Deze representatie wordt vervolgens gebruikt door een classificatiemethode die het een klasselable geeft.

\npar \cite{gesture-FNN-HMM} stelt een gebaarherkenningssyteem voor die zich focust op de handen. Uit het dieptebeeld van een Kinectcamera wordt de hand gesegmenteerd via thresholding. Drie features worden vervolgens bijgehouden en gebruikt: de verandering van de handvorm, de beweging van de hand in het tweedimensionaal vlak en de beweging van de hand in de diepte (z-as).
\\Er wordt gebruik gemaakt van twee classificatie methodes: Hidden Markov Modellen (HMM) en Fuzzy Neurale Netwerken (FNN). HMM is een classificatietechniek die rekening houdt met het tijdsaspect. FNN is een combinatie van fuzzy (vage) theorie en artifici\"ele neurale netwerken.

\subsection{Grammaticale samenstelling}
TODO
%Camera
%Features
%
%Gebaarsegmentatie
%Waar begint een gebaar en waar eindigt het
%Gebaarherkenning
%Welk gebaar en wat betekent het
%Boodschap ontleden --> gebaren samenstellen, grammatica, (bv. letterwoorden, kleur)

\section{One-shot learning}

\subsection{Uitbreidbaarheid herkenningssysteem}
Een taal is voortdurend in verandering.
Het lexicon van een gebarentaal groeit mee met de tijd. Gloednieuwe termen of zaken die voordien geen beschrijving kenden in een gebarentaal worden toegevoegd. Hogescholen en universiteiten gebruikten lange tijd geen gebarentaal waardoor er weinig wetenschappelijke termen opgenomen zijn in de gebarenschat. Gelukkig komen er vandaag steeds meer wetenschappelijke gebaren bij.

\npar Een automatische herkenningssysteem zal moeten leren omgaan met dit groeiende lexicon. Een strategie kan zijn om na verloop van tijd (vanaf een bepaald aantal nieuwe gebaren) het systeem te hertrainen met voorbeelden van de oude gebaren en de nieuwe gebaren. Hierbij wordt er dus vanaf nul gestart en een nieuw model opgebouwd.

\npar Een eerste probleem is het verzamelen van de data. Deep learning methodieken hebben een complexe structuur en erg veel parameters. Om een grote hoeveelheid parameters te optimaliseren voor een taak heb je een grote hoeveelheid data nodig om uit te leren. Als we dus een nieuw gebaar willen bijleren aan een herkenningssysteem hebben we vele voorbeelden nodig van dit ene gebaar, liefst tegen verschillende achtergronden, uitgevoerd door verschillende personen en in verschillende lichtomstandigheden.
\\ Het maken van dergelijke datasets is een erg kostelijke en tijdrovende opdracht.

\npar Het model vanaf nul terug hertrainen vraagt veel tijd en rekenvermogen. Alle vooraf opgedane kennis wordt gewist dus alle tijd en moeite die eerder ge\"investeerd werd is voor niets. Het systeem zal ook minstens evenveel rekentijd nodig hebben als tijdens de opbouw van het vorige model.

\npar Als we zo een aantal keer het herkenningssysteem willen uitbreiden zullen we veel kostbare tijd en energie verspillen.

\subsection{Bijleren bij mensen}
TODO

\subsection{One-shot learning in de literatuur}
\npar \cite{oneshot-vis-concepts} stelt een generatief model voor voor het herkennen van handgeschreven karakters. Het vertrekt vanuit de notie dat de mens een teken schrijft in verschillende halen of lijnen en ook zo een nieuw teken leert herkennen. Er wordt een dataset opgebouwd van 1600 karakters die door verschillende gebruikers online geregistreerd worden. Elke lijn die een gebruiker plaatst wordt opgeslaan alsook de volgorde van tekenen. Zo bestaat elk teken uit een opeenvolging van lijnen met verschillende vorm en lengte. De verzameling van al deze lijnen wordt gebruikt als voorafgaande kennis om nieuwe tekens bij te leren met een voorbeeld. Het nieuwe teken wordt door het systeem opgedeeld in lijncomponenten die dan afgetoetst worden tegen het model. Zo ontstaat een nieuwe representatie voor het bijgeleerde gebaar die kan gebruikt worden voor herkenning. Er wordt een nauwkeurigheid van 54.9 \% behaald tegenover 39.6 \% voor een implementatie aan de hand van Deep Boltzman Machines (DBM). Wanneer bij het aanleren van het nieuwe gebaar de lijninformatie van de dataset wordt gebruikt in plaats van die van het systeem zelf wordt een nauwkeurigheid van 63.7 \% waargenomen.

\npar \cite{oneshot-gesture-rgbd} buigt zich over de ChaLearn One-shot Learning Gesture Challenge 2011 en leert vanuit slechts een voorbeeld een gebaar te herkennen zonder enige voorgaande kennis. Er wordt ge\"experimenteerd met een aantal feature descriptors en classificatie methodes waaruit Extended Motion History Images (Extended MHI) en Maximum Correlation Co\"effici\"ent (MCC) als best presterende worden gevonden. Extended MHI  bestaat zelf uit drie representaties: MHI en Inversed recording (INV) focussen zich op bewegingsinformatie respectievelijk in het begin en op het einde van het gebaar terwijl Gait Energy Information (GEI) repetitieve beweging registreert. Het systeem behaalt een Levensteihnafstand van 0.29685 (tussen 0 en 1 waarbij 0 optimaal) op de validatieset en presteert zeer goed op gebaren waarin er veel beweging is. De twee meer statische gebaren uit de dataset worden het minst goed gededecteerd met een nauwkeurigheid lager dan 45 \%.

\npar \cite{oneshot-video-segm} stelt een convolutioneel neuraal netwerk (CNN) voor die uit een voorbeeld de voorgrond van de achtergrond onderscheidt in een video. Het CNN wordt vooraf getraind op de ImageNet dataset. Een dataset van 1,2 miljoen afbeeldingen uit meer dan duizend categori\"en. Door deze pre-training op een zeer ruime dataset is het model algemeen en leert het eigenlijk wat 'een object' is. Hierna wordt het model verfijnd voor het volgen van een voorgrondsobject uit een video. Het eerste frame van de video wordt gemaskeerd en hierop stelt het model zich af. Deze architectuur verbetert de state-of-the-art op de Densely Annotated Video Segmentation (DAVIS) dataset met 11.2 \% (79.8\% vs 68.0\%).

\section{Doelstelling}
%Dit onderzoek gaat na of we effici\"ent gebruik kunnen maken van vooraf opgedane kennis in een convolutioneel neuraal netwerk 







 