\chapter{Inleiding}
\section{Gebarentaal}

Gebarentaal is in de eerste plaats een taal. Taal is een begrip dat moeilijk te defini\"eren valt en de meeste pogingen hiertoe beperken zich tot gesproken taal. Een definitie die ook voor gebarentaal kan gebruikt worden is: een natuurlijk ontstaan communicatiemidddel waarmee je kan communiceren over alles wat je denkt, ziet, voelt en droomt. \cite{bron}

\npar Gesproken taal en gebarentaal verschillen in de manier waarop gecommuniceerd wordt: oraal-auditief tegenover gestueel-visueel. Door middel van hand-, hoofd- en armbewegigingen wordt een woord "uitgesproken" en vervolgens visueel waargenomen.

\npar Een gebarentaal onstaat, net zoals een gesproken taal, spontaan en natuurlijk door contact tussen mensen. Net door deze spontane ontwikkeling is er geen universele gebarentaal. Evenals we verschillende gesproken talen en dialecten kennen per land of regio zijn er ook verschillende gebarentalen \cite{bron-mothi}. In Nederland is er bijvoorbeeld de Nederlandse Gebarentaal (NGT) en in Belgi\"e de Vlaamse Gebarentaal (VGT) en de Waalse Gebarentaal (la Langue des Signes de Belgique Francophone, LSFB).

\npar Een gebarentaal heeft een eigen grammatica en lexicon. Het lexicon of de gebarenschat is de verzameling van alle woorden of gebaren in de taal. Het lokale gebarenschat moet volledig onafhankelijk vna het lokale woordenschat worden beschouwd.
\\Bepaalde woorden uit de ene taal kunnen niet eenduidig vertaald worden in een andere taal. Het woord "gezelligheid" kent bijvoorbeeld geen Engelse vertaling en voor het Duitse "fingerspitzengef\"uhl" hebben we in de Nederlandse taal ook geen alternatief.
\\Tussen een gebarentaal en een gesproken taal geldt dezelfde verhouding. 


Visuele taal
hand, hoofd, lichaamsbewegingen
eigen woordenschat en grammatica
een volwaardige taal
geen een op een vertaling van woord naar gebaar

regionale verschillen, vlaamse gebarentaal, nederlandse gebarentaal
zijn er uit zichzelf gekomen, pogingen tot standaardisatie
vele varianten
moeilijke communicatie onderling

communicatie met horenden via schrift, 
\section{Automatische gebarentaalherkenning}
\subsection{Gebarenherkenning}
\subsection{Gebarensegmentatie}
\section{One-shot learning}
\section{Doelstelling}