\chapter{Besluit en toekomstperspectieven}

\npar In dit werk werd een model vooropgesteld dat, met behulp van radio-over-fiber en het moving cell concept, een belangrijk probleem van celgebaseerde oplossingen, namelijk het handoverprobleem, oplost.  Door dit probleem van de baan te helpen, is een belangrijke hinderpaal voor het verwezenlijken van een breedbandtoegangsnetwerk voor snel bewegende gebruikers weggenomen.  De conceptuele werking werd aangetoond met behulp van simulaties.

\npar Dit werk is echter slechts het topje van de ijsberg, en van een ijsberg is geweten dat de grootste hoeveelheid zich onder water bevindt.  Het werk is dan ook nog lang niet af.  Om een beter zicht te hebben op het geheel, zijn betere metingen nodig.  In eerste instantie denken we aan emulaties, met echte draadloze kaarten, waar de effecten van de verschillende signalen op elkaar en hun omgeving beter in kaart gebracht kan worden.  Een mogelijkheid hiervoor is de FAMOUS-opstelling, echter deze houdt nog steeds met verschillende parameters geen rekening, zoals bijvoorbeeld het doppler-effect.  Ook de aanwezigheid van de hoogspanningslijnen bij treinen kan een rol spelen, om nog maar te zwijgen van het effect in tunnels.  Een emulatie met een echte beweging is dus zeker gewenst.  Verder dienen ook nog een heleboel optimalisaties te gebeuren, die ingegeven kunnen worden door dergelijke emulaties.  Een voorbeeld is het moment dat de centrale wisselt van RAU.  Bij een simulatie bekomen we steeds dezelfde waarden voor de signaalsterkte, en zal het signaal van zodra we het bereik van een RAU binnentreden, alleen maar verbeteren.  In realiteit is de signaalsterkte geen stabiel gegeven.  Een eenmalige opstoot van een signaal kan de centrale doen overschakelen terwijl het signaal op die plaats helemaal nog niet sterk genoeg is data op een goede manier te transporteren.  Een mogelijkheid kan hier zijn om bij het oppikken van het eerste signaal nog een bepaalde tijdspanne te wachten alvorens over te schakelen.  Ook andere gegevens kunnen geoptimaliseerd worden, zoals de keuze van de antennes.  In onze simulaties werd gebruik gemaakt van een omnidirectionele antenne die in alle richtingen een even sterk signaal uitzendt.  Met richtantennes kan dat signaal sterk verbeterd worden, wat een impact heeft op het bereik ervan.

\npar En zelfs als dergelijke experimenten uitgevoerd zijn en betere parameterwaarden opgeleverd hebben, is de kous nog lang niet af.  Het capteren van een volledige frequentieband met radio-over-fiber en het nadien reconstrueren van de individuele kanalen is immers geen evidente zaak.  Ook hier dient nog heel wat onderzoekswerk verricht te worden vooraleer remote antenna units op een effici�nte manier gebouwd kunnen worden.

\npar De hamvraag is echter: is het sop de kool wel waard?  Conceptueel en technisch kunnen grenzen doorbroken worden, maar wat wil de consument hiervoor betalen?  De fabricage van speciaal voor deze doeleinden ontworpen RAU's, de installatie van vele RAU's met stroomvoorziening en dergelijke, de glasvezelkabel, de centrales,...  Zaken waaraan zeker geen laag kostenplaatje hangt, hoe eenvoudig de RAU's ook mogen ontworpen zijn.  Is de consument wel bereid hiervoor genoeg te betalen om de investering rendabel te maken?  Ook dit dient onderzocht te worden.

\npar We kunnen dan ook besluiten dat er technische grenzen doorbroken zijn of kunnen worden, het is conceptueel haalbaar, maar er dient nog veel onderzoek verricht te worden in dit onderwerp.  Onderzoek dat niet alleen voer is voor ingenieurs, maar ook voor economen.  De toekomst ziet er echter bijzonder rooskleurig uit.